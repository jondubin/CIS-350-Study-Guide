% Created 2015-05-01 Fri 12:03
\documentclass[11pt]{article}
\usepackage[utf8]{inputenc}
\usepackage[T1]{fontenc}
\usepackage{fixltx2e}
\usepackage{graphicx}
\usepackage{longtable}
\usepackage{float}
\usepackage{wrapfig}
\usepackage{rotating}
\usepackage[normalem]{ulem}
\usepackage{amsmath}
\usepackage{textcomp}
\usepackage{marvosym}
\usepackage{wasysym}
\usepackage{amssymb}
\usepackage{hyperref}
\tolerance=1000
\author{Reed Rosenbluth}
\date{\today}
\title{350 Final}
\hypersetup{
  pdfkeywords={},
  pdfsubject={},
  pdfcreator={Emacs 24.4.1 (Org mode 8.2.10)}}
\begin{document}

\maketitle
\tableofcontents


\section{CIS 350 Final}
\label{sec-1}

75 point exam. 120 minutes. Double sided study sheet

\section{First Half (25\%)}
\label{sec-2}
\subsection{{\bfseries\sffamily TODO} software process models}
\label{sec-2-1}
\subsection{{\bfseries\sffamily TODO} agile}
\label{sec-2-2}
\subsection{{\bfseries\sffamily TODO} software configuration management}
\label{sec-2-3}
\subsection{{\bfseries\sffamily TODO} requirements gathering and coumentation}
\label{sec-2-4}
\subsection{{\bfseries\sffamily TODO} requirement analysis}
\label{sec-2-5}
\subsection{{\bfseries\sffamily TODO} software architecture}
\label{sec-2-6}
\subsection{{\bfseries\sffamily TODO} design concepts}
\label{sec-2-7}
\subsection{{\bfseries\sffamily TODO} design patterns}
\label{sec-2-8}

\section{Second Half (75\%)}
\label{sec-3}
\subsection{{\bfseries\sffamily DONE} usability and UI}
\label{sec-3-1}
\begin{itemize}
\item Why is usability important?
\begin{itemize}
\item \textbf{Therac-25} case study
\begin{itemize}
\item Radiation therapy machine involved in at least 6 accidents between 1985-1987
\item patients were give massive overdoses of radiation
\item It moved some safety features from hardware to software
\item Software from the older version (Therac-20) was reused and was assumed to be workign correctly
\item \textbf{Hamilton, Ontario, July 1985}
\begin{itemize}
\item During treatment, machine stopped and reported "no does"
\item This was a common occurence, so the technician pressed "P" to proceed and re-deliver the dose
\item This happened four more times, after which the machine went into suspend mode
\item Patient received 5x dose and died four months later
\end{itemize}
\item In Tyler Texas (1986) a technician reported "Malfunction 54" which did not exist in the user manuel
\begin{itemize}
\item the operator had requested 202 units, but only 6 were delivered and the patient died five months later
\item the manufacturer could not reproduce this bug so the machine was put back into service
\item Malfunction 54 happened again five weeks later and another patient died
\end{itemize}
\end{itemize}
\end{itemize}
\item Definition of usability: \textbf{Five E's}
\begin{itemize}
\item Effective
\item Efficient
\item Engaging
\item Error Tolerant
\item Easy to learn
\end{itemize}
\item User-centered design
\begin{itemize}
\item needs, wants, and limitations of end users are given extensive attention
\end{itemize}
\item Information visualization and metaphors
\item Measuring usability: studies, heuristics, metrics
\begin{itemize}
\item \textbf{Heuristics}
\begin{itemize}
\item general guidelines or widely accepted "best practices"
\item Nielsen's usability heuristics
\begin{itemize}
\item Match between system and real world (skeumorphism)
\item Consistency and standards
\item Help and documentation
\item User control and freedom
\item visibility of system status
\item Flexibility and efficiency of use
\item error prevention
\item recognition rather than recall
\item recognize/diagnose erros
\item aesthetic and minimalist design
\end{itemize}
\end{itemize}
\item \textbf{Usability studies}
\begin{itemize}
\item observe people using the system under normal circumstances
\item methods
\begin{itemize}
\item surveys
\item focus groups
\item ethnography (field observation)
\item observation in controlled setting
\end{itemize}
\end{itemize}
\item \textbf{Metrics}
\begin{itemize}
\item measure quantitative aspects such as time to complete a task, error rate, memorability, etc.
\item Task analysis
\begin{itemize}
\item \textbf{Hierarchical Task Analysis (HTA)}
\begin{itemize}
\item task is broken down into goals and subgoals
\end{itemize}
\item \textbf{Cognitative Task Analysis (CTA)}
\begin{itemize}
\item also includes cognitative study (time spent thinking about what to do)
\end{itemize}
\end{itemize}
\end{itemize}
\item \textbf{Human Reliability Assessment}
\begin{itemize}
\item error rate: how often does the user make a mistake
\item cognitive load: how much can the user keep in his mind during a task?
\item memorability: how much does the user remember?
\end{itemize}
\end{itemize}
\end{itemize}

\subsection{{\bfseries\sffamily DONE} integration}
\label{sec-3-2}
\begin{itemize}
\item individual software modules are combined and \textbf{tested as a group}. occurs after unit testing
\item \textbf{Big-bang integration}
\begin{itemize}
\item write everything and hope
\item all components are integrated simultaneously, after which everything ist ested as a whole
\end{itemize}
\item \textbf{Top-down integration}
\begin{itemize}
\item testing conducted from main module to sub module. If the sub module is not developed then it is replaced by a temporary program called a stub
\end{itemize}
\item \textbf{Bottom-up integration}
\begin{itemize}
\item lowest level components are tested first.
\item they are then used to facilitate the testing of higher level components
\end{itemize}
\end{itemize}
\subsection{{\bfseries\sffamily DONE} test driven development and defensive programming}
\label{sec-3-3}
\begin{itemize}
\item \textbf{Test Driven Development}
\begin{itemize}
\item write test cases \textbf{first}
\item then write minimum amount of code to pass the test
\item finally refactor the code to acceptable standards
\end{itemize}
\item \textbf{Defensive Programing}
\begin{itemize}
\item \emph{McConnell, Code Complete, ch. 8, 23, 25-26}
\begin{itemize}
\item the recognition that programs will have problems and modifications, and that a smart programmer will develop code accordingly.
\item Protecting program from \textbf{invalid inputs}
\begin{itemize}
\item check the values of all data from external sources
\item check the values of all routine input parameters
\item decide how to handle bad inputs
\end{itemize}
\item \textbf{Assertions}
\begin{itemize}
\item code used during development that allows a program to check itself as it runs
\item use error-handling code for conditions you expect to occur, use assertions for conditions that should never occur
\end{itemize}
\item \textbf{Error handling}
\begin{itemize}
\item many ways to do this\ldots{} return a neutral value (like 0), substitute the next piece of valid data, return the same answer as the previous time, substitute the closest legal value, log a warning message to a file, return an error code, etc.
\end{itemize}
\item \textbf{Exceptions}
\begin{itemize}
\item a specific means by which code can pass along errors or exceptional events to teh cade that called it
\item try/catch in java
\end{itemize}
\end{itemize}
\item \emph{Hunt \& Thomas, Pragmatic Programmer, ch. 4}
\begin{itemize}
\item Not only do pragmatic programmers not trust other people's code, they don't trust theirs either.
\item \textbf{Design by Contract}
\begin{itemize}
\item documenting the rights and responsibilities of software modules to ensure program correctness.
\item \textbf{Preconditions}: a routines requirements
\item \textbf{Postconditions}: what the routine is guaranteed to do; the state of the world when the routine is done
\item \textbf{Class invariants}: A class ensures that this condition is always true from teh perspective of the caller.
\end{itemize}
\item Implementing DBC
\begin{itemize}
\item Can partially emulate contracts with \textbf{assertions}, but assertions don't propogate down an inheritance hierarchy.
\item Some languages have built in support for DBC (like Eiffel)
\end{itemize}
\item Use exceptions rarely, for unexpected events.
\end{itemize}
\end{itemize}
\end{itemize}
\subsection{{\bfseries\sffamily DONE} analyzability (readability, understandability)}
\label{sec-3-4}
\begin{itemize}
\item \emph{Boswell, The Art of Readable Code, ch. 2-3}
\begin{itemize}
\item \textbf{Packing Information into Names}
\begin{itemize}
\item \textbf{choose specific words}. \texttt{Size()} vs \texttt{Height()}, \texttt{NumNodes()}, \texttt{MemoryBytes()}
\item \textbf{Avoid generic names} like \texttt{tmp} and =retval=* unless there is a reason
\item \textbf{Prefer concrete names over abstract names} - \texttt{ServerCanStart()} vs \texttt{CanListenOnPort()}
\item \textbf{Attaching extra information to a name}
\begin{itemize}
\item Values with units: \texttt{delay} vs \texttt{delay\_secs}
\item other examples
\begin{itemize}
\item \texttt{password} vs \texttt{plaintext\_password}
\item \texttt{html} vs \texttt{html\_utf8}
\end{itemize}
\end{itemize}
\item \textbf{Use longer names for larger scopes}
\item \textbf{Use capitalization, underscores, etc. in a meaningful way}
\end{itemize}
\item \textbf{The best names} are ones that can't be misconstrued
\end{itemize}
\end{itemize}
\subsection{{\bfseries\sffamily TODO} concurrency \& synchronization}
\label{sec-3-5}
\begin{itemize}
\item Early computers could only run one program at a time
\item \textbf{Multi-programming} operating systems (1970s) made it appear as if multiple programs were running simultaneously
\item \textbf{Instruction-Level Parallelism}
\begin{itemize}
\item rather than wait for an instruction to finish, start the next one asap
\item \textbf{pipelining}: fetch/decode one instruction while executing another
\end{itemize}
\item \textbf{Thread-Level Parallelism}
\begin{itemize}
\item now we have threads, which allow programs to do more than one thing at a time
\item \textbf{In Java}
\begin{itemize}
\item each thread has it's own stack and program counter
\item threads in the same process share a heap and static variables
\item Defining and starting a Thread
\begin{verbatim}
public class HelloRunnable implements Runnable {

    public void run() {
        System.out.println("Hello from a thread!");
    }

    public static void main(String args[]) {
        (new Thread(new HelloRunnable())).start();
    }
}
\end{verbatim}
\item 
\end{itemize}
\end{itemize}
\end{itemize}
\subsection{{\bfseries\sffamily TODO} efficiency and performance}
\label{sec-3-6}
\subsection{{\bfseries\sffamily TODO} code smells and refactoring}
\label{sec-3-7}
\subsection{{\bfseries\sffamily TODO} testing basics}
\label{sec-3-8}
\subsection{{\bfseries\sffamily TODO} black-box testing}
\label{sec-3-9}
\subsection{{\bfseries\sffamily TODO} white-box testing}
\label{sec-3-10}
\subsection{{\bfseries\sffamily TODO} debugging}
\label{sec-3-11}

\section{Reading assignments from 2nd half}
\label{sec-4}
\subsection{Boswell, The Art of Readable Code, ch. 2-3}
\label{sec-4-1}
\subsection{McConnell, Code Complete, ch. 8, 23, 25-26}
\label{sec-4-2}
\subsection{Hunt \& Thomas, Pragmatic Programmer, ch. 4}
\label{sec-4-3}
\subsection{Fowler, Refactoring, ch. 2-3}
\label{sec-4-4}
\subsection{Braude, Software Engineering, ch. 24 and 26}
\label{sec-4-5}
\subsection{Ammann \& Offutt, Software Testing, ch. 1}
\label{sec-4-6}
\subsection{Jorgensen, Software Testing, ch. 5-6}
\label{sec-4-7}
% Emacs 24.4.1 (Org mode 8.2.10)
\end{document}